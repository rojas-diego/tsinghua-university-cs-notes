\documentclass[12pt]{article}

%Packages%
\usepackage{amsmath, bm, parskip}
\usepackage{hyperref}

%Meta%
\title{Introductory Combinatorics - Permutations and Combinations}
\author{Diego ROJAS}

%Document%
\begin{document}

\maketitle

\tableofcontents

%Introduction%	
\section{Four Basic Counting Principles}

\subsection{The Addition Principle}

Suppose that a set $S$ is partitioned into pairwise disjoint parts $S_1,S_2, \dots ,S_m$. The number of objects in $S$ can be determined by finding the number of objects in each of the parts and adding the numbers obtained.
%
\begin{equation}
\begin{gathered}
	S = S_1 \cup S_2 \cup \cdots \cup S_m \\
	|S| = |S_1| + |S_2| + \cdots + |S_m|
\end{gathered}
\end{equation}
%
The number of objects of a set $S$ is denoted by $|S|$ and is called the size of $S$.

\subsection{The Multiplication Principle}

Let $S$ be a set of ordered pairs $(a, b)$ of objects, where the first object $a$ comes from a set of size $p$, and for each choice of object $a$ there are $q$ choices for object $b$. Then the size of $S$ is $p \times q$.
\\
It's important for $a$ and $b$ to be independent choices.


\subsection{The Substraction Principle}

Let $A$ be a set and let $U$ be a larger set containing $A$. Let
%
$$
\bar{A} = U \backslash A = \{ x \in U : x \notin A \}
$$
%
be the \textit{complement} of $A$. Then the number of objects in $A$ is given by:
%
$$
|A| = |U| - |\bar{A}|
$$

\subsection{The Division Principle}

Let $S$ be a finite set that is partitioned into $k$ parts in such a way that each part contains the same number of objects. Then the number of parts in the partition is given by the rule:

$$
k = \frac{|S|}{\text{number of objects in a part}}
$$

\section{Permutations of Sets}

We denote $P(n, r)$ the number of \textit{r-permutations} of a set of $n$ elements.

\textbf{Linear permutations}: the number of \textit{linear} $r$-permutations of a set of $n$ elements where $n \geq r$ is given by:
%
\begin{equation}
\begin{gathered}
P(n, r) = n \times (n - 1) \times \cdots \times (n - r + 1) = P(n, r) = \frac{n!}{(n - r)!} \\
P(n, n) = n!
\end{gathered}
\end{equation}

\textbf{Circular permutations}: the number of \textit{circular} $r$-permutations of a set of $n$ elements where $n \geq r$ is given by
%
\begin{equation}
\frac{P(n, r)}{r} = \frac{n!}{r \cdot (n - r)!}, \ P(n, n) =  (n - 1)!
\end{equation}

\section{Combinations (Subsets) of Sets}

\textbf{Combination:} a \textit{combination} or \textit{subset} of a set $S$ denotes an unordered selection of the elements of $S$.
%
We denote by $\binom n r$ the number of $r$-subsets of a set $S$ of size $n$.
%
\begin{equation}
\binom n r = C(n, r) = \frac{n!}{r! (n - r)!}
\end{equation}
%
It also stands that
%
\begin{equation}
	\binom{n}{r} = \binom{n}{n - r}
\end{equation}


\textbf{Pascal's formula:} for all integers $n$ and $k$ with $1 \leq k \leq n - 1$:
%
\begin{equation}
\binom n k = \binom{n - 1}{k} + \binom{n - 1}{k - 1}
\end{equation}

\textbf{Theorem:} for $n > 0$:
%
\begin{equation}
\binom n 0 + \binom n 1 + \cdots + \binom n n = 2^n
\end{equation}

\section{Permutations of Multisets}

\textbf{Infinite repetition:} the number of $r$-permutations of $k$ distinct objects, each available in unlimited supply (or if $r \leq \text{supply}$), equals $k^r$.

\textbf{Finite repetition:} for a set $S$ with k elements repeated $n_1, n_2, \cdots n_k$ times, the number of permutations is given by
%
\begin{equation}
	\frac{n!}{n_1!n_2! \cdots n_k!}
\end{equation}
%
Note there are no easy formula for the case $n < r$.

\section{Combinations of Multisets}

\textbf{Infinite repetition:} Let $S$ be a multiset with objects of $k$ types, each with an infinite repetition number (or at least $r$). Then the number of $r$-combinations of $S$ equals
%
\begin{equation}
	\binom{r + k - 1}{r} = \binom{r + k - 1}{k - 1}
\end{equation}

\section{Finite Probability}

\end{document}