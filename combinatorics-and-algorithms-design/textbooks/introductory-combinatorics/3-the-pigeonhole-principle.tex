\documentclass[12pt]{article}

%Packages%
\usepackage{amsmath, bm, parskip, hyperref}

%Meta%
\title{Introductory Combinatorics - The Pigeonhole Principle}
\author{Diego ROJAS}

%Document%
\begin{document}

\maketitle
\tableofcontents

\section{Simple Form}

\textbf{Pigeonhole principle:} If $n+1$ objects are distributed into $n$ boxes, then at least one box has two or more elements in it.

\section{Strong Form}

\textbf{Pigeonhole principle:} Let $q_1, q_2, q_n$ be positive integers. If
%
\begin{equation}
	q_1 + q_2 + \cdots + q_n - n + 1
\end{equation}
%
objects are distributed into $n$ boxes, then the first box contains at least $q_1$ objects, or the second box contains at least $q_2$ objects, ..., or the $n^{th}$ box contains at least $q_n$ objects.

\section{A Theorem of Ramsey}

\end{document}